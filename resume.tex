%!TEX TS-program = xelatex
%!TEX encoding = UTF-8 Unicode
% Awesome CV LaTeX Template
%
% This template has been downloaded from:
% https://github.com/posquit0/Awesome-CV
%
%


%%%%%%%%%%%%%%%%%%%%%%%%%%%%%%%%%%%%%%
%     Configuration
%%%%%%%%%%%%%%%%%%%%%%%%%%%%%%%%%%%%%%
%%% Themes: Awesome-CV
\documentclass[]{awesome-cv}
\usepackage{textcomp}
%%% Override a directory location for fonts(default: 'fonts/')
\fontdir[fonts/]

%%% Configure a directory location for sections
\newcommand*{\sectiondir}{resume/}

%%% Override color
% Awesome Colors: awesome-emerald, awesome-skyblue, awesome-red, awesome-pink, awesome-orange
%                 awesome-nephritis, awesome-concrete, awesome-darknight
%% Color for highlight
% Define your custom color if you don't like awesome colors
\colorlet{awesome}{awesome-red}
%\definecolor{awesome}{HTML}{CA63A8}
%% Colors for text
%\definecolor{darktext}{HTML}{414141}
%\definecolor{text}{HTML}{414141}
%\definecolor{graytext}{HTML}{414141}
%\definecolor{lighttext}{HTML}{414141}

%%% Override a separator for social informations in header(default: ' | ')
%\headersocialsep[\quad\textbar\quad]
    \begin{document}
    
%%%%%%%%%%%%%%%%%%%%%%%%%%%%%%%%%%%%%%
%     Profile
%%%%%%%%%%%%%%%%%%%%%%%%%%%%%%%%%%%%%%
\begin{center}
	\headerfirstnamestyle{Rehan} \headerlastnamestyle{Guha} \\
	\vspace{2mm}
	{\faEnvelope\ rehanguha29@gmail.com} | {\faMobile\ (+91) 9874223189} | {\faMapMarker\ India} | {\faLink\ https://rehanguha.github.io/}
\end{center}
%%%%%%%%%%%%%%%%%%%%%%%%%%%%%%%%%%%%%%
%     Experience
%%%%%%%%%%%%%%%%%%%%%%%%%%%%%%%%%%%%%%
\cvsection{Work Experience}
\begin{cventries}
	\cventry
	{Machine Learning researcher}
	{Pramti}
	{Chennai, Imdia}
	{201 – 201}
	{\begin{cvitems}
		\item {R\textquotesingle{}n\textquotesingle{}D}
		\item {Managmeent}
		\end{cvitems}}
	\cventry
	{ASE}
	{Accenture}
	{Chennai, India}
	{321 – 231}
	{\begin{cvitems}
		\item {Python}
		\item {Quartz}
		\end{cvitems}}
\end{cventries}
\cvsection{Honors \& Awards}
\begin{cvhonors}
	\cvhonor
	{Runners-Up}
	{In this challenge, we use a drone to search for and locate a soil probe. The drone picks up the probe and takes it to a drop-off location. Then, the drone must autonomously return to a rover and land in its trunk. This was accomplished using Rtabmap, a SLAM package used to estimate the position and velocity states of the rover with visual odometry and our own tailor-made Kalman Filter conflated with GPS based state estimation and the visual odometry based states.}
	{National Science Foundation (NSF) \& Arizona State University (ASU)}
	{Jul 2020}
	\cvhonor
	{Accenture Celebrates Excellence}
	{Category - Innovation; Award type - Team}
	{Accenture}
	{Aug 2018}
	\cvhonor
	{Accenture Celebrates Excellence}
	{Category - Innovation; Award type - Individual}
	{Accenture}
	{May 2018}
	\cvhonor
	{Accenture Celebrates Excellence}
	{Category - Client and Customer; Award type - Team}
	{Accenture}
	{Nov 2017}
	\cvhonor
	{Student Performance Award}
	{Received the award for excellence innovation and performace.}
	{Institute of Engineering \& Management, Kolkata}
	{May 2015}
\end{cvhonors}
%%%%%%%%%%%%%%%%%%%%%%%%%%%%%%%%%%%%%%
%     Education
%%%%%%%%%%%%%%%%%%%%%%%%%%%%%%%%%%%%%%
\cvsection{Education}
\begin{cventries}
	\cventry
	{Deep Learning (AICTE approved FDP course)}
	{Indian Institute of Technology, Kharagpur}
	{}
	{Jul 2019 – Oct 2019}
	{}
	\cventry
	{Bachelor of Computer Application (B.C.A.) in Computer Applications}
	{Institute of Engineering \& Management, Kolkata}
	{Kolkata, India}
	{2013 – 2016}
	{GPA: 8.5}
	\cventry
	{All India Senior School Certificate Examination in Computer Science}
	{Delhi Public School, Ruby Park}
	{Kolkata, India}
	{2011 – 2013}
	{}
\end{cventries}

\vspace{-2mm}
\cvsection{Skills}
\begin{cventries}
	\cventry
	{}
	{\def\arraystretch{1.15}{\begin{tabular}{ l l }
		Machine Learning:  & {\skill{ Imbalance Classification, Statistical Analysis}} \\
		Deep Learning:  & {\skill{ CNN}} \\
		\end{tabular}}}
	{}
	{}
	{}
\end{cventries}

\vspace{-7mm}
\cvsection{Projects}
\begin{cventries}
	\cventry
	{ProejctDescription1}
	{Project1}
	{Python, Angular JS}
	{github link}
	{}
	
	\vspace{-5mm}
	\cventry
	{ProjectDescription2}
	{Project2}
	{Deep Leanring, ML}
	{Github2}
	{}
	
	\vspace{-5mm}
\end{cventries}
\ 
\end{document}